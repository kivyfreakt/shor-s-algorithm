\anonsection{Приложение A}

\begin{center}
\textbf{Исходный код программы}
\end{center}

\begin{verbatim}
import numpy as np
# Importing standard Qiskit libraries
from qiskit import QuantumCircuit, QuantumRegister, ClassicalRegister, execute, circuit, IBMQ
from qiskit.tools.jupyter import *
from qiskit.visualization import *
from ibm_quantum_widgets import *

# Loading your IBM Quantum account(s)
provider = IBMQ.load_account()

from fractions import Fraction
from math import gcd

def c_amod15(power):
    '''
        Преобразование a^2^power mod 15
        :param a: Случайное число, взаимно простое с N
        :param power: Степень
        :return: Вентиль преобразования
    '''
    U = QuantumCircuit(4)
    for iteration in range(power):
            U.swap(2,3)
            U.swap(1,2)
            U.swap(0,1)
            for q in range(4):
                U.x(q)
    U = U.to_gate()
    U.name = "7^%i mod 15" % power
    c_U = U.control()
    return c_U

def inversed_qft(n):
    '''
        Обратное квантовое преобразование фурье
        :param n: Количество входов квантового вентиля
        :return: Вентиль обратного преобразования фурье
    '''
    qc = QuantumCircuit(n)
    for qubit in range(n//2):
        qc.swap(qubit, n-qubit-1)
    for j in range(n):
        for m in range(j):
            qc.cp(-np.pi/float(2**(j-m)), m, j)
        qc.h(j)
    qc.name = "QFT-1"
    return qc

def shor_period(a, N):
    '''
        Квантовый алгоритм Шора для нахождения периода функции a^x mod N
        :param a: Параметр функции a
        :param N: Параметр функции N
    '''
    n_count = 8

    qc = QuantumCircuit(QuantumRegister(8,'x'),QuantumRegister(4,'f(x)'),ClassicalRegister(8))

    # инициализация расчетных кубитов в состоянии |+>
    for q in range(n_count):
        qc.h(q)

    # вспомогательные регистры в состоянии |1>
    qc.x(3+n_count)

    # выполнение controlled-U операции
    for q in range(n_count):
        qc.append(c_amod15(2**q), [q] + [i+n_count for i in range(4)])

    # выполнение QFT-1 операции
    qc.append(inversed_qft(n_count), range(n_count))

    # измерение рассчетных кубитов
    qc.measure(range(n_count), range(n_count))

    # вывод схемы на экран
    print(qc)

    # симуляция
    qasm_sim = Aer.get_backend('qasm_simulator')
    t_qc = transpile(qc, qasm_sim)
    obj = assemble(t_qc, shots=2048)
    result = qasm_sim.run(obj, memory=True).result()
    readings = result.get_memory()

    # вычисление периода
    phase = int(readings[0], 2)/(2**n_count)
    frac = Fraction(phase).limit_denominator(N)
    r = frac.denominator
    return r


def factorization(N):
    '''
        Алгоритм факторизации чисел (поиск простых делителей числа)
        :param N: Число, необходимое факторизовать
    '''
    a = 7
    factor_found = False
    attempt = 0
    while not factor_found:
        attempt += 1
        print("\nПопытка %i:" % attempt)
        phase = shor_period(a, N)
        frac = Fraction(phase).limit_denominator(N)
        r = frac.denominator
        print("Период: r = %i" % r)
        if phase != 0:
            guesses = [gcd(a**(r//2)-1, N), gcd(a**(r//2)+1, N)]
            for guess in guesses:
                # проверка того, что угаданное число действительно является делителем
                if guess not in [1, N] and (N % guess) == 0:
                    print('Простые делители числа', N, ':', guess)
                    factor_found = True
                else:
                    factor_found = False

N = 15
factorization(N)
\end{verbatim}

\newpage

