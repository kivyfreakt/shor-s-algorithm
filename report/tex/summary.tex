\clearpage

\begin{center}
\textbf{Аннотация}
\end{center}

	Цель практического задания – добиться понимания принципа работы одного из квантовых алгоритмов и осуществить его реализацию на эмуляторе квантовой системы. В моём случае необходимо проработать квантовый алгоритм факторизации чисел, придуманный в 1994 году Питером Шором.
	
	Для достижения этой цели мне было необходимо усвоить принципы квантовых вычислений, изучить понятия кубита, системы кубитов, и операций над ними. Затем перейти к изучению квантовых алгоритмов и их отличий от классических. Финальным этапом работы стала реализация алгоритма Шора.

%\begin{center}
%\textbf{SUMMARY}
%\end{center}

%Briefly (8-10 lines) to describe the the purpose and main contents of the practice work.


\newpage
